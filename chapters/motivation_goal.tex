\chapter{Motivation \& Outline}
\label{ch:motivation_goal}

\section{Motivation}

The overarching purpose of this master's thesis is to outline the first steps taken towards building a generally reusable, automatic interpretation and hypothesis generation machine.

While many analyses, including those previously referenced, can successfully aid scientists in interpretation of data, each are developed with specific data sets, knowledge assemblies, or application scenarios in mind. Those that were developed using schemata that lacked the generalized multi-scale and multi-modal (schema-free) integration enabled by BEL could have potentially disregarded relevant and important knowledge. 

\section{Outline}

The first section of this thesis describes the PyBEL, the framework built to parse and manipulate BEL Script and resulting knowledge assemblies. The following section describes the development the Bio2BEL data integration framework and the beginning of a cross-scale data integration project similar to Pathway Commons. The final section describes the development of algorithms for analyzing the robustness of knowledge assemblies, preprocessing techniques, and ultimately, proposes a reusable, general, schema-free analytical technique that generates hypotheses with knowledge-driven analyses of data.
